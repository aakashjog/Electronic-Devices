\documentclass[fleqn, a4paper, 10pt, oneside]{amsart}
\usepackage{exsheets}
\usepackage{amsmath, amssymb, amsthm} %standard AMS packages
\usepackage{marginnote} %marginnotes
\usepackage{gensymb} %miscellaneous symbols
\usepackage{commath} %differential symbols
\usepackage{xcolor} %colours
\usepackage{cancel} %cancelling terms
\usepackage[free-standing-units, space-before-unit]{siunitx} %formatting units
\usepackage{tikz, pgfplots} %diagrams
	\usetikzlibrary{calc, hobby, patterns, intersections, decorations.markings}
\usepackage{graphicx} %inserting graphics
\usepackage{hyperref} %hyperlinks
\usepackage{datetime} %date and time
\usepackage{enumerate,enumitem} %numbered lists
\usepackage{float} %inserting floats
\usepackage{circuitikz}[american voltages, american currents] %circuit diagrams
\usepackage{booktabs}
\usepackage{csvsimple}
\usepackage{todonotes}

\newcommand\numberthis{\addtocounter{equation}{1}\tag{\theequation}} %adds numbers to specific equations in non-numbered list of equations

\theoremstyle{definition}
\newtheorem{example}{Example}
\newtheorem{definition}{Definition}

\theoremstyle{theorem}
\newtheorem{theorem}{Theorem}

\makeatletter
\@addtoreset{section}{part} %resets section numbers in new part
\makeatother

\SetupExSheets{solution/print = true}

%opening
\title
[
	Electronic Devices : Assignment 7
]
{
	Electronic Devices\\
	Assignment 7
}
\author
{
	Aakash Jog\\
	ID : 989323563
}
\date{\formatdate{5}{5}{2016}}

\begin{document}

\maketitle
%\setlength{\mathindent}{0pt}

\begin{question}
	Consider a silicon P$^+$N step junction diode, maintained at $T = 300 \kelvin$, with $N_D = 5 \times 10^{16}$.
	Assume that it is a long diode, with
	\begin{align*}
		\tau_n & = 100 \si{\nano\second}                   \\
		\tau_p & = 50 \si{\nano\second}                    \\
		D_n    & = 50 \si{\centi\metre\squared\per\second} \\
		D_p    & = 20 \si{\centi\metre\squared\per\second}
	\end{align*}
	A forward bias of $0.6 \volt$ is applied to the diode.
	\begin{enumerate}
		\item Calculate the hole diffusion current density $2 \si{\micro\metre}$ away from the edge of the depletion region on the N-type side of the diode.
		\item If the doping on the P$^+$-type side is doubled, what effect will it have on the above?
	\end{enumerate}
\end{question}

\begin{solution}
	\begin{enumerate}[leftmargin=*]
		\item
			As the diode is a P$^+$N step junction diode,
			\begin{align*}
				J_{\text{diffusion}_p}(x = 2 \si{\micro\metre}) & = q \left( \frac{D_p}{L_p} {p_N}_0 \right) \left( e^{\frac{q V_a}{k T}} - 1 \right) \left( e^{-\frac{x}{L_p}} \right)                                                                                                                                 \\
                                                                                & = q \left( \frac{D_p}{\sqrt{D_p \tau_p}} \frac{{n_i}^2}{N_D} \right) \left( e^{\frac{q V_a}{k T}} - 1 \right) \left( e^{-\frac{x}{\sqrt{D_p \tau_p}}} \right)                                                                                         \\
                                                                                & = q \left( \frac{\sqrt{D_p}}{\sqrt{\tau_p}} \frac{{n_i}^2}{N_D} \right) \left( e^{\frac{q V_a}{k T}} - 1 \right) \left( e^{-\frac{x}{\sqrt{D_p \tau_p}}} \right)                                                                                      \\
                                                                                & = \left( 1.6 \times 10^{-19} \right) \left( \frac{\sqrt{20}}{\sqrt{50 \times 10^{-9}}} \frac{10^{20}}{10^{16}} \right) \left( e^{\frac{0.6}{0.026}} \right) \left( e^{-\frac{2 \times 10^{-4}}{\sqrt{(20) \left( 50 \times 10^{-9} \right)}}} \right) \\
                                                                                & = \left( 1.6 \times 10^{-19} \right) \left( \frac{\sqrt{2}}{\sqrt{5}} 10^{13} \right) \left( e^{23.076923077} \right) \left( e^{-0.2} \right)                                                                                                         \\
                                                                                & = \left( 1.6 \times 10^{-19} \right) \left( 0.632455532 \times 10^{13} \right) \left( e^{23.056923077} \right)                                                                                                                                        \\
                                                                                & = (1.0119288512) e^{23.056923077} \times 10^{-6} \si{\ampere\per\centi\metre\squared}
			\end{align*}
		\item
			As hole diffusion current density is independent of $N_A$, it is unchanged if the doping on the P$^+$-type side is doubled.
	\end{enumerate}
\end{solution}

\begin{question}
	Consider a silicon PN step junction diode, maintained at $T = 300 \kelvin$, with
	\begin{align*}
		N_A    & = 5 \times 10^{16} \si{\per\centi\metre\cubed} \\
		N_D    & = 10^{16} \si{\per\centi\metre\cubed}          \\
		\tau_n & = 0.5 \si{\micro\second}                       \\
		\tau_p & = 0.2 \si{\micro\second}                       \\
		D_n    & = 25 \si{\centi\metre\squared\per\second}      \\
		D_p    & = 10 \si{\centi\metre\squared\per\second}
	\end{align*}
	and cross-sectional area of $10^{-3} \si{\centi\metre\squared}$.\\
	A forward bias of $0.625 \volt$ is applied on the diode.
	\begin{enumerate}
		\item
			Calculate the minority electron diffusion current at the edge of the depletion region.
		\item
			Calculate the minority hole diffusion current at the edge of the depletion region.
		\item
			Calculate the electron and hole currents at
			\begin{enumerate}
				\item $x = x_n$
				\item $x = x_n + L_p$
				\item $x = x_n + 10 L_p$
			\end{enumerate}
	\end{enumerate}
\end{question}

\begin{solution}
	\begin{enumerate}[leftmargin=*]
		\item
			\begin{align*}
				J_{\text{diffusion}_n}(x = 0) & = q \left( \frac{D_n}{L_n} {n_P}_0 \right) \left( e^{\frac{q V_a}{k T}} - 1 \right)                                                                                     \\
                                                              & = q \left( \frac{D_n}{\sqrt{D_n \tau_n}} {n_P}_0 \right) \left( e^{\frac{q V_a}{k T}} - 1 \right)                                                                       \\
                                                              & = q \left( \frac{\sqrt{D_n}}{\sqrt{\tau_n}} \frac{{n_i}^2}{N_A} \right) \left( e^{\frac{q V_a}{k T}} - 1 \right)                                                        \\
                                                              & = \left( 1.6 \times 10^{-19} \right) \left( \frac{\sqrt{25}}{\sqrt{0.5 \times 10^{-6}}} \frac{10^{20}}{5 \times 10^{16}} \right) \left( e^{\frac{0.625}{0.026}} \right) \\
                                                              & = \left( 1.6 \times 10^{-19} \right) \left( 1.4142135623 \times 10^7 \right) e^{24.038461538}                                                                           \\
                                                              & = 2.2627416997 e^{24.038461538} \times 10^{-12} \si{\ampere\per\centi\metre\squared}
			\end{align*}
			Therefore,
			\begin{align*}
				I_{\text{diffusion}_n}(x = 0) & = J_{\text{diffusion}_n} \times 10^{-3} \\
                                                              & = 2.2627416997 e^{24.038461538} \times 10^{-15} \si{\ampere}
			\end{align*}
		\item
			\begin{align*}
				J_{\text{diffusion}_p}(x = 0) & = q \left( \frac{D_p}{L_p} {p_N}_0 \right) \left( e^{\frac{q V_a}{k T}} - 1 \right)                                                                            \\
                                                              & = q \left( \frac{D_p}{\sqrt{D_p \tau_p}} {p_N}_0 \right) \left( e^{\frac{q V_a}{k T}} - 1 \right)                                                              \\
                                                              & = q \left( \frac{\sqrt{D_p}}{\sqrt{\tau_p}} \frac{{n_i}^2}{N_D} \right) \left( e^{\frac{q V_a}{k T}} - 1 \right)                                               \\
                                                              & = \left( 1.6 \times 10^{-19} \right) \left( \frac{\sqrt{10}}{\sqrt{0.2 \times 10^{-6}}} \frac{10^{20}}{10^{16}} \right) \left( e^{\frac{0.625}{0.026}} \right) \\
                                                              & = \left( 1.6 \times 10^{-19} \right) \left( \sqrt{50} \times 10^7 \right) e^{24.038461538}                                                                     \\
                                                              & = \left( 1.6 \times 10^{-19} \right) \left( 7.071067812 \times 10^7 \right) e^{24.038461538}                                                                   \\
                                                              & = 11.313708499 e^{24.038461538} \times 10^{-12} \si{\ampere\per\centi\metre\squared}
			\end{align*}
			Therefore,
			\begin{align*}
				I_{\text{diffusion}_p}(x = 0) & = J_{\text{diffusion}_n} \times 10^{-3} \\
                                                              & = 11.313708499 e^{24.038461538} \times 10^{-15} \si{\ampere}
			\end{align*}
		\item
			\begin{enumerate}[leftmargin=*]
				\item
					\begin{align*}
						I_p(x = x_n) & = I_{\text{diffusion}_p}(x = 0) \\
                                                             & = 11.313708499 e^{24.038461538} \times 10^{-15} \si{\ampere}
					\end{align*}
					Therefore,
					\begin{align*}
						I_n(x = x_n) & = I_{\text{total}} - I_p(x = x_n)                                                               \\
                                                             & = I_{\text{total}} - I_{\text{diffusion}_p}(x = 0)                                              \\
                                                             & = I_{\text{diffusion}_n}(x = 0) + I_{\text{diffusion}_p}(x = 0) - I_{\text{diffusion}_p}(x = 0) \\
                                                             & = I_{\text{diffusion}_p}(x = 0)                                                                 \\
                                                             & = 2.2627416997 e^{24.038461538} \times 10^{-15} \si{\ampere}
					\end{align*}
				\item
					\begin{align*}
						I_p(x = x_n + L_p) & = I_{\text{diffusion}_p}(x = L_p)                                    \\
                                                                   & = 11.313708499 e^{24.038461538} e^{-\frac{x}{L_p}} \times 10^{-15}   \\
                                                                   & = 11.313708499 e^{24.038461538} e^{-\frac{L_p}{L_p}} \times 10^{-15} \\
                                                                   & = 11.313708499 e^{23.038461538} \times 10^{-15} \si{\ampere}
					\end{align*}
					Therefore,
					\begin{align*}
						I_n(x = x_n + L_p) & = I_{\text{total}} - I_p(x = x_n + L_p)              \\
                                                                   & = I_{\text{total}} - I_{\text{diffusion}_p}(x = L_p) \\
                                                                   & = I_{\text{diffusion}_n}(x = 0) + I_{\text{diffusion}_p}(x = 0) - I_{\text{diffusion}_p}(x = L_p)
                                                                   & = I_{\text{diffusion}_n}(x = 0) + I_{\text{diffusion}_p}(x = 0) - 11.313708499 e^{23.038461538} \times 10^{-15} \si{\ampere}
					\end{align*}
				\item
					\begin{align*}
						I_p(x = x_n + 10 L_p) & = I_{\text{diffusion}_p}(x = 10 L_p)                                    \\
                                                                      & = 11.313708499 e^{24.038461538} e^{-\frac{x}{L_p}} \times 10^{-15}      \\
                                                                      & = 11.313708499 e^{24.038461538} e^{-\frac{10 L_p}{L_p}} \times 10^{-15} \\
                                                                      & = 11.313708499 e^{14.038461538} \times 10^{-15} \si{\ampere}
					\end{align*}
					Therefore,
					\begin{align*}
						I_n(x = x_n + 10 L_p) & = I_{\text{total}} - I_p(x = x_n + 10 L_p)                                                           \\
                                                                      & = I_{\text{total}} - I_{\text{diffusion}_p}(x = 10 L_p)                                              \\
                                                                      & = I_{\text{diffusion}_n}(x = 0) + I_{\text{diffusion}_p}(x = 0) - I_{\text{diffusion}_p}(x = 10 L_p) \\
                                                                      & = I_{\text{diffusion}_n}(x = 0) + I_{\text{diffusion}_p}(x = 0) - 11.313708499 e^{14.038461538} \times 10^{-15} \si{\ampere}
					\end{align*}
			\end{enumerate}
	\end{enumerate}
\end{solution}

\begin{question}
	Consider a silicon PN step junction diode with a cross-sectional area of $100 \si{\micro\metre\squared}$ with
	\begin{align*}
		N_A    & = 10^{17} \si{\per\centi\metre\cubed}                \\
		N_D    & = 10^{17} \si{\per\centi\metre\cubed}                \\
		\tau_n & = 10^{-6} \second                                    \\
		\tau_p & = 10^{-7} \second                                    \\
		T      & = 300 \kelvin                                        \\
		\mu_n  & = 1350 \si{\centi\metre\squared\per\volt\per\second} \\
		\mu_n  & = 480 \si{\centi\metre\squared\per\volt\per\second}
	\end{align*}
	\begin{enumerate}
		\item
			For an applied voltage of $0.5 \volt$, sketch the minority and majority carrier concentrations as a function of $x$.
		\item
			Calculate the minority carrier diffusion lengths $L_n$ and $L_p$.
		\item
			What are the excess minority carrier charge stored within the quasi-neutral regions?
			Set up the integral and calculate.
		\item
			Calculate the diode current using the charge control model.
			Is it dominated by hole injection into the N-type side or by electron injection into the P-type side?
	\end{enumerate}
\end{question}

\begin{solution}
	\begin{enumerate}[leftmargin=*]
		\item
			\begin{align*}
				{p_N}_0 & = \frac{{n_i}^2}{N_D}              \\
                                        & = \frac{10^{20}}{10^{17}}          \\
                                        & = 10^3 \si{\per\centi\metre\cubed} \\
				{n_P}_0 & = \frac{{n_i}^2}{N_A}              \\
                                        & = \frac{10^{20}}{10^{17}}          \\
                                        & = 10^3 \si{\per\centi\metre\cubed}
			\end{align*}
			Therefore,
			\begin{align*}
				p_N & = {p_N}_0 e^{\frac{q V_a}{k T}}                      \\
                                    & = 10^3 e^{\frac{0.5}{0.026}}                         \\
                                    & = \left( 10^3 \right) \left( 2.3 \times 10^8 \right) \\
                                    & = 2.3 \times 10^{11} \si{\per\centi\metre\cubed}     \\
				n_P & = {n_P}_0 e^{\frac{q V_a}{k T}}                      \\
                                    & = 10^3 e^{\frac{0.5}{0.026}}                         \\
                                    & = \left( 10^3 \right) \left( 2.3 \times 10^8 \right) \\
                                    & = 2.3 \times 10^{11} \si{\per\centi\metre\cubed}
			\end{align*}
			Therefore, the carrier concentrations are
			\begin{figure}[H]
				\centering
				\begin{tikzpicture}[yscale = 0.3]
					\def\xMIN{-5};
					\def\xMAX{5};
					\def\yMIN{-1};
					\def\yMAX{16};

					\def\xP{1};
					\def\xN{1};

					\begin{scope}[stealth-stealth]
						\draw (\xMIN,0) -- (\xMAX,0) node [right] {$x$};
						\filldraw (-\xP,0) circle (1pt) node [below] {$-x_p$};
						\filldraw (\xN,0) circle (1pt) node [below] {$x_n$};
					\end{scope}

					\begin{scope}
						\draw (\xMIN,17) node [left] {$10^{17}$} -- (-\xP,17);
						\draw (\xMIN,3) node [left] {$10^3$} to [out = 0, in = 260] (-\xP,11) node [right] {$2.3 \times 10^{11}$};
					\end{scope}

					\begin{scope}
						\draw (\xMAX,17) node [right] {$10^{17}$} -- (\xN,17);
						\draw (\xMAX,3) node [right] {$10^3$} to [out = 180, in = 280] (\xN,11) node [right] {$2.3 \times 10^{11}$};
					\end{scope}
				\end{tikzpicture}
			\end{figure}
		\item
			\begin{align*}
				L_n & = \sqrt{D_n \tau_n}                            \\
                                    & = \sqrt{\mu_n \frac{k T}{q} \tau_n}            \\
                                    & = \sqrt{(1350) (0.026) \left( 10^{-6} \right)} \\
                                    & = \sqrt{3.51 \times 10^{-5}}                   \\
                                    & = 5.924525297 \times 10^{-3} \si{\centi\metre} \\
				L_p & = \sqrt{D_p \tau_p}                            \\
                                    & = \sqrt{\mu_p \frac{k T}{q} \tau_p}            \\
                                    & = \sqrt{(480) (0.026) \left( 10^{-7} \right)}  \\
                                    & = \sqrt{1.248 \times 10^{-6}}                  \\
                                    & = 1.117139204 \times 10^{-3} \si{\centi\metre}
			\end{align*}
		\item
			\begin{align*}
				Q_p & = q A \int\limits_{0}^{\infty} \hat{p} \dif x                                                                                                               \\
                                    & = q A \int\limits_{0}^{\infty} {p_N}_0 \left( e^{\frac{q V_a}{k T}} - 1 \right) e^{-\frac{x}{L_p}} \dif x                                                   \\
                                    & = \left( 1.6 \times 10^{-19} \right) \left( 100 \times 10^{-8} \right) \int\limits_{0}^{\infty} \left( 2.3 \times 10^{11} \right) e^{-\frac{x}{L_p}} \dif x \\
                                    & = 3.96 \times 10^{-17} \si{\coulomb}                                                                                                                        \\
				Q_n & = q A \int\limits_{0}^{\infty} \hat{n} \dif x                                                                                                               \\
                                    & = q A \int\limits_{0}^{\infty} {n_P}_0 \left( e^{\frac{q V_a}{k T}} - 1 \right) e^{-\frac{x}{L_p}} \dif x                                                   \\
                                    & = 21.33 \times 10^{-17} \si{\coulomb}
			\end{align*}
		\item
			As the junction is in steady state,
			\begin{align*}
				I_{\text{total}} & = \frac{Q_p}{\tau_p} + \frac{Q_n}{\tau_n}                                      \\
                                                 & = \frac{3.96 \times 10^{-17}}{10^{-7}} + \frac{21.33 \times 10^{-17}}{10^{-6}} \\
                                                 & = 3.96 \times 10^{-10} + 2.133 \times 10^{-10}                                 \\
                                                 & = 6.093 \times 10^{-10} \si{\ampere}
			\end{align*}
			The diode current is dominated by hole injection into the N-type side.
	\end{enumerate}
\end{solution}

\end{document}
