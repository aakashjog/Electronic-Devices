\documentclass[fleqn, a4paper, 10pt, oneside]{amsart}
\usepackage{exsheets}
\usepackage{amsmath, amssymb, amsthm} %standard AMS packages
\usepackage{marginnote} %marginnotes
\usepackage{gensymb} %miscellaneous symbols
\usepackage{commath} %differential symbols
\usepackage{xcolor} %colours
\usepackage{cancel} %cancelling terms
\usepackage[free-standing-units, space-before-unit]{siunitx} %formatting units
\usepackage{tikz, pgfplots} %diagrams
	\usetikzlibrary{calc, hobby, patterns, intersections, decorations.markings}
\usepackage{graphicx} %inserting graphics
\usepackage{hyperref} %hyperlinks
\usepackage{datetime} %date and time
\usepackage{enumerate,enumitem} %numbered lists
\usepackage{float} %inserting floats
\usepackage{circuitikz}[american voltages, american currents] %circuit diagrams
\usepackage{booktabs}
\usepackage{csvsimple}
\usepackage{todonotes}

\newcommand\numberthis{\addtocounter{equation}{1}\tag{\theequation}} %adds numbers to specific equations in non-numbered list of equations

\theoremstyle{definition}
\newtheorem{example}{Example}
\newtheorem{definition}{Definition}

\theoremstyle{theorem}
\newtheorem{theorem}{Theorem}

\makeatletter
\@addtoreset{section}{part} %resets section numbers in new part
\makeatother

\SetupExSheets{solution/print = true}

%opening
\title
[
	Electronic Devices : Assignment 6
]
{
	Electronic Devices\\
	Assignment 6
}
\author
{
	Aakash Jog\\
	ID : 989323563
}
\date{\formatdate{20}{4}{2016}}

\begin{document}

\maketitle
%\setlength{\mathindent}{0pt}

\begin{question}
	Consider a PN step junction at $300 \kelvin$ with
	\begin{align*}
		N_A &= 10^{16} \si{\per\centi\metre\cubed}\\
		N_D &= 10^{16} \si{\per\centi\metre\cubed}\\
		\tau_n &= 0.5 \si{\micro\second}\\
		\tau_p &= 0.1 \si{\micro\second}\\
		D_n &= 25 \si{\centi\metre\squared\per\second}\\
		D_p &= 10 \si{\centi\metre\squared\per\second}
	\end{align*}
	Assume that a reverse bias of $5 \volt$ is applied.
	The junction is uniformly illuminated such that
	\begin{align*}
		G_{\text{optical}} &= 10^{21} \si{\per\centi\metre\cubed\per\second}
	\end{align*}
	\begin{enumerate}
		\item
			Calculate the photocurrent $J_{\text{optical}}$ in the junction.
		\item
			Calculate the total current density in the junction.
		\item
			Explain the biasing conditions for this illuminated PN junction for operation as a solar cell.
			Explain why using a PiN junction, instead of the basic PN structure can improve the device performance for solar cells.
	\end{enumerate}
\end{question}

\begin{solution}
	\begin{enumerate}[leftmargin=*]
		\item
			\begin{align*}
				V_{\text{BI}} &= \frac{k T}{q} \ln\left( \frac{N_A N_D}{{n_i}^2} \right)\\
				&= 0.026 \ln\left( \frac{\left( 10^{16} \right) \left( 10^{16} \right)}{10^{20}} \right)\\
				&= 0.026 \ln\left( 10^{12} \right)\\
				&= (0.026) (27.63102112)\\
				&= 0.7184065491 \volt
			\end{align*}
			Therefore,
			\begin{align*}
				W &= \sqrt{\frac{2 \varepsilon \varepsilon_0 (V_{\text{BI}} - V_a)}{q} \left( \frac{N_A + N_D}{N_A N_D} \right)}\\
				&= \sqrt{\frac{(2) (11.8) \left( 8.85 \times 10^{-14} \right) (0.72 + 5)}{1.6 \times 10^{-19}} \left( \frac{2 \times 10^{16}}{10^{32}} \right)}\\
				&= \sqrt{\frac{(2) (11.8) \left( 8.85 \times 10^{-14} \right) (5.72)}{1.6 \times 10^{-19}} \left( \frac{2 \times 10^{16}}{10^{32}} \right)}\\
				&= 0.000122203 \si{\centi\metre}\\
				&= 1.2 \times 10^{-4} \si{\centi\metre}
			\end{align*}
			Also,
			\begin{align*}
				L_n &= \sqrt{D_n \tau_n}\\
				&= \sqrt{(25) \left( 0.5 \times 10^{-6} \right)}\\
				&= 3.54 \times 10^{-3}\\
				L_p &= \sqrt{D_p \tau_p}\\
				&= \sqrt{(10) \left( 0.1 \times 10^{-6} \right)}\\
				&= 1 \times 10^{-3}
			\end{align*}
			Therefore,
			\begin{align*}
				J_{\text{optical}} &= -q G_{\text{optical}} \left( L_n + L_p + W \right)\\
				&= -\left( 1.6 \times 10^{-19} \right) \left( 10^{21} \right) \left( 3.54 \times 10^{-3} + 10^{-3} + 1.2 \times 10^{-4} \right)\\
				&= -\left( 1.6 \times 10^{-19} \right) \left( 10^{21} \right) \left( 4.66 \times 10^{-3} \right)\\
				&= -0.7456 \si{\ampere\per\centi\metre\squared}
			\end{align*}
		\item
			\begin{align*}
				J_{\text{total}} &= q \left( \frac{D_p}{L_p} {p_N}_0 +  \frac{D_n}{L_n} {n_P}_0 \right) \left( e^{\frac{q V_a}{k T}} - 1 \right) + J_{\text{optical}}\\
				&= -q \left( \frac{D_p}{L_p} {p_N}_0 +  \frac{D_n}{L_n} {n_P}_0 \right) + J_{\text{optical}}\\
				&= -q \left( \frac{D_p}{L_p} \frac{{n_i}^2}{N_D} +  \frac{D_n}{L_n} \frac{{n_i}^2}{N_D} \right) + J_{\text{optical}}\\
				&= -\left( 1.6 \times 10^{-19} \right) \left( \frac{10}{10^{-3}} \frac{10^{20}}{10^{16}} + \frac{25}{3.54 \times 10^{-3}} \frac{10^{20}}{10^{16}} \right) -0.7456\\
				&= -\left( 1.6 \times 10^{-19} \right) \left( 10^{8} + 7.1 \times 10^{7} \right) - 0.7456\\
				&= -27.36 \times 10^{-12} - 0.7456\\
				&\approx -0.7456 \si{\ampere\per\centi\metre\squared}
			\end{align*}
		\item
			As the total current is negative, the junction works as a solar cell.
			If the junction is not in reverse bias, the current is not negative, and hence the junction does not function as a solar cell.\\
			For a PiN junction, the width of the depletion region is larger.
			Hence, the photocurrent is greater than that for a PN junction.
			Therefore, a PiN junction is better than a PN junction for use as a solar cell.
	\end{enumerate}
\end{solution}

\end{document}
