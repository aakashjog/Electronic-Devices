\documentclass[fleqn, a4paper, 10pt, oneside]{amsart}
\usepackage{exsheets}
\usepackage{amsmath, amssymb, amsthm} %standard AMS packages
\usepackage{marginnote} %marginnotes
\usepackage{gensymb} %miscellaneous symbols
\usepackage{commath} %differential symbols
\usepackage{xcolor} %colours
\usepackage{cancel} %cancelling terms
\usepackage[free-standing-units, space-before-unit]{siunitx} %formatting units
\usepackage{tikz, pgfplots} %diagrams
	\usetikzlibrary{calc, hobby, patterns, intersections, decorations.markings}
\usepackage{graphicx} %inserting graphics
\usepackage{hyperref} %hyperlinks
\usepackage{datetime} %date and time
\usepackage{enumerate,enumitem} %numbered lists
\usepackage{float} %inserting floats
\usepackage{circuitikz}[american voltages, american currents] %circuit diagrams
\usepackage{booktabs}
\usepackage{csvsimple}
\usepackage{todonotes}

\newcommand\numberthis{\addtocounter{equation}{1}\tag{\theequation}} %adds numbers to specific equations in non-numbered list of equations

\theoremstyle{definition}
\newtheorem{example}{Example}
\newtheorem{definition}{Definition}

\theoremstyle{theorem}
\newtheorem{theorem}{Theorem}

\makeatletter
\@addtoreset{section}{part} %resets section numbers in new part
\makeatother

\SetupExSheets{solution/print = true}

%opening
\title
[
	Electronic Devices : Assignment 4
]
{
	Electronic Devices\\
	Assignment 4
}
\author
{
	Aakash Jog\\
	ID : 989323563
}
\date{\formatdate{12}{4}{2016}}

\begin{document}

\maketitle
%\setlength{\mathindent}{0pt}

\begin{question}
	Consider a NP junction at room temperature, with voltage $V$ applied across the junction such that
	\begin{align*}
		\mu_n &= 1000 \si{\centi\metre\squared\per\volt\per\second}\\
		\mu_p &= 100 \si{\centi\metre\squared\per\volt\per\second}\\
		L_n &= 10^{-4} \si{\centi\metre}\\
		L_p &= 10^{-4} \si{\centi\metre}
	\end{align*}
	Assume that both the $N$ and $P$ regions are long and the depletion region is long.
	The figure below shows the equilibrium carrier densities with respect to the position $x$ for the NP junction in equilibrium.
	\begin{enumerate}
		\item
			Is this NP junction made of silicon?
			If it is not silicon, is it made of a material with higher or lower energy band gap than silicon?
		\item
			Redraw the equilibrium carrier densities for the NP junction, but implementing the depletion approximation.
		\item
			For a forward bias of $V_a = 0.2 \volt$, what is the electron concentration at the edge of the depletion region on the P-type side?
		\item
			For a forward bias of $V_a = 0.2 \volt$, what is the electron current density at the edge of the depletion region on the P-type side?
		\item
			For a forward bias of $V_a = 0.2 \volt$, draw minority carrier profile as a function of $x$.
			Include the values at the edge of the depletion region and far away from the junction.
		\item
			For a forward bias of $V_a = -0.2 \volt$, draw minority carrier profile as a function of $x$.
			Include the values at the edge of the depletion region and far away from the junction.
	\end{enumerate}
\end{question}

\begin{solution}
	\begin{enumerate}[leftmargin=*]
		\item
			\begin{align*}
				{n_i}^2(x_n) &= p_0(x_n) n_0(x_n)\\
				&= 10^9 10^{17}\\
				&= 10^{26}\\
				\therefore n_i &= 10^{13}
			\end{align*}
			Therefore, the junction is not made of silicon.\\
			As the intrinsic carrier concentration is higher than that for silicon, the energy band gap of the material is lower than that of silicon.
		\item
			Assuming there is no generation and recombination of EHPs in the depletion region, the carrier concentration changes linearly and not exponentially.
			\begin{figure}[H]
				\centering
				\begin{tikzpicture}[yscale = 0.5]
					\def\xMIN{-5};
					\def\xMAX{5};
					\def\yMIN{0};
					\def\yMAX{18};
					\def\xP{1};
					\def\xN{1};

					\begin{scope}[stealth-stealth]
						\draw (\xMIN,0) -- (\xMAX,0) node [right] {$x$};
						\draw [-stealth] (-\xN,\yMIN) -- (-\xN,\yMAX) node [above left] {$\log\left[ n_0,p_0 \right]$};
						\draw [-stealth] (\xP,\yMIN) -- (\xP,\yMAX) node [above right] {$\log\left[ n_0,p_0 \right]$};
						\filldraw (-\xN,0) circle (1pt) node [below] {$-x_n$};
						\filldraw (\xP,0) circle (1pt) node [below] {$x_p$};
						\foreach \y in {\yMIN,...,\yMAX}
						{
							\filldraw (-\xN,\y) circle (1pt) node [right] {$\y$};
						}

						\foreach \y in {\yMIN,...,\yMAX}
						{
							\filldraw (\xP,\y) circle (1pt) node [left] {$\y$};
						}
					\end{scope}

					\begin{scope}
						\draw (\xMIN,9) node [left] {$p_0(x_n)$} -- (-\xN,9);
						\draw (\xP,15) -- (\xMAX,15) node [right] {$p_0(x_p)$};
						\draw (\xMIN,17) node [left] {$n_0(x_n)$} -- (-\xN,17);
						\draw (\xP,11) -- (\xMAX,11) node [right] {$n_0(x_p)$};
					\end{scope}
				\end{tikzpicture}
			\end{figure}
		\item
			\begin{align*}
				n_N &= {n_N}_0 e^{\frac{q V_a}{k T}}\\
				&= 10^{11} e^{\frac{0.2}{0.026}}\\
				&= 2191.425868 \times 10^{11} \si{\per\centi\metre\cubed}\\
				&= 2.191425868 \times 10^{14} \si{\per\centi\metre\cubed}
			\end{align*}
		\item
			\begin{align*}
				J_{n}\left( 0^+ \right) &= q \left( \frac{D_n}{L_n} {n_P}_0 \right) \left( e^{\frac{q V_a}{k T}} - 1 \right)\\
				&= q \left( \frac{\mu_n k T}{L_n} {n_P}_0 \right) \left( e^{\frac{q V_a}{k T}} - 1 \right)\\
				&= 9.2 \si{\ampere\per\centi\metre\squared}
			\end{align*}
		\item
			\begin{align*}
				p_N &= {p_N}_0 e^{\frac{q V_a}{k T}}\\
				&= 10^9 e^{\frac{0.2}{0.026}}\\
				&= 2.2 \times 10^{12}\\
				n_P &= {n_P}_0 e^{\frac{q V_a}{k T}}\\
				&= 10^{11} e^{\frac{0.2}{0.026}}\\
				&= 2.2 \times 10^{14}
			\end{align*}
			\begin{figure}[H]
				\centering
				\begin{tikzpicture}[yscale = 0.5]
					\def\xMIN{-5};
					\def\xMAX{5};
					\def\yMIN{-5};
					\def\yMAX{5};
					\def\xP{1};
					\def\xN{1};

					\begin{scope}[stealth-stealth]
						\draw (\xMIN,0) -- (\xMAX,0) node [right] {$x$};
						\filldraw (-\xN,0) circle (1pt) node [below] {$-x_n$};
						\filldraw (\xP,0) circle (1pt) node [below] {$x_p$};
					\end{scope}

					\begin{scope}
						\draw (\xMIN,9) node [left] {$10^9$} to [out = 0, in = 260] (-\xN,12) node [right] {$2.2 \times 10^3$};
					\end{scope}

					\begin{scope}
						\draw (\xMAX,11) node [right] {$10^{11}$} to [out = 180, in = 280] (\xP,14) node [left] {$2.2 \times 10^{14}$};
					\end{scope}
				\end{tikzpicture}
			\end{figure}
		\item
			\begin{align*}
				p_N &= {p_N}_0 e^{\frac{q V_a}{k T}}\\
				&= 10^9 e^{-\frac{0.2}{0.026}}\\
				&= 4.6 \times 10^5\\
				n_P &= {n_P}_0 e^{\frac{q V_a}{k T}}\\
				&= 10^{11} e^{-\frac{0.2}{0.026}}\\
				&= 4.6 \times 10^7
			\end{align*}
			\begin{figure}[H]
				\centering
				\begin{tikzpicture}[yscale = 0.5]
					\def\xMIN{-5};
					\def\xMAX{5};
					\def\yMIN{-5};
					\def\yMAX{5};
					\def\xP{1};
					\def\xN{1};

					\begin{scope}[stealth-stealth]
						\draw (\xMIN,0) -- (\xMAX,0) node [right] {$x$};
						\filldraw (-\xN,0) circle (1pt) node [below] {$-x_n$};
						\filldraw (\xP,0) circle (1pt) node [below] {$x_p$};
					\end{scope}

					\begin{scope}
						\draw (\xMIN,9) node [left] {$10^9$} to [out = 0, in = 100] (-\xN,5) node [right] {$4.6 \times 10^5$};
					\end{scope}

					\begin{scope}
						\draw (\xMAX,11) node [right] {$10^{11}$} to [out = 180, in = 80] (\xP,7) node [left] {$4.6 \times 10^7$};
					\end{scope}
				\end{tikzpicture}
			\end{figure}
	\end{enumerate}
\end{solution}

\end{document}
